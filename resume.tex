% This work may be distributed and/or modified under the
% conditions of the LaTeX Project Public License version 1.3c,
% available at http://www.latex-project.org/lppl/.


\documentclass[11pt,a4paper,sans]{moderncv}        % possible options include font size ('10pt', '11pt' and '12pt'), paper size ('a4paper', 'letterpaper', 'a5paper', 'legalpaper', 'executivepaper' and 'landscape') and font family ('sans' and 'roman')

% modern themes
\moderncvstyle{banking}                            % style options are 'casual' (default), 'classic', 'oldstyle' and 'banking'
\moderncvcolor{blue}                                % color options 'blue' (default), 'orange', 'green', 'red', 'purple', 'grey' and 'black'
%\renewcommand{\familydefault}{\sfdefault}         % to set the default font; use '\sfdefault' for the default sans serif font, '\rmdefault' for the default roman one, or any tex font name
%\nopagenumbers{}                                  % uncomment to suppress automatic page numbering for CVs longer than one page

% character encoding
\usepackage[utf8]{inputenc}                       % if you are not using xelatex ou lualatex, replace by the encoding you are using
%\usepackage{CJKutf8}                              % if you need to use CJK to typeset your resume in Chinese, Japanese or Korean

% adjust the page margins
\usepackage[scale=0.75, top=1in, bottom=.5in]{geometry}
%\setlength{\hintscolumnwidth}{3cm}                % if you want to change the width of the column with the dates
%\setlength{\makecvtitlenamewidth}{10cm}           % for the 'classic' style, if you want to force the width allocated to your name and avoid line breaks. be careful though, the length is normally calculated to avoid any overlap with your personal info; use this at your own typographical risks...

\usepackage{import}
%\renewcommand*{\httplink}[2][]{%
%  \ifthenelse{\equal{#1}{}}%
%    {\href{#2}{#2}}%
%    {\href{#2}{#1}}}
% personal data
\name{Amisha}{Budhiraja}
\title{Resume Computer Science Engineer}                               % optional, remove / comment the line if not wanted
\address{\#26-D, New Kitchlu Nagar, Hambran Road, Ludhiana, Punjab - 141008}{}{}% optional, remove / comment the line if not wanted; the "postcode city" and and "country" arguments can be omitted or provided empty
\phone[mobile]{+91 8360478192}                   % optional, remove / comment the line if not wanted
%\phone[fixed]{01234 123456}                    % optional, remove / comment the line if not wanted
%\phone[fax]{+3~(456)~789~012}                      % optional, remove / comment the line if not wanted
\email{amishabudhiraja96@gmail.com}                               % optional, remove / comment the line if not wanted
\social [github][github.com/amisha2016]{github.com/amisha2016}
\homepage {amisha2016.wordpress.com}                         % optional, remove / comment the line if not wanted
%\extrainfo{additional information}                 % optional, remove / comment the line if not wanted
%\photo[64pt][0.4pt]{picture}                       % optional, remove / comment the line if not wanted; '64pt' is the height the picture must be resized to, 0.4pt is the thickness of the frame around it (put it to 0pt for no frame) and 'picture' is the name of the picture file
%\quote{Some quote}                                 % optional, remove / comment the line if not wanted

% to show numerical labels in the bibliography (default is to show no labels); only useful if you make citations in your resume
%\makeatletter
%\renewcommand*{\bibliographyitemlabel}{\@biblabel{\arabic{enumiv}}}
%\makeatother
%\renewcommand*{\bibliographyitemlabel}{[\arabic{enumiv}]}% CONSIDER REPLACING THE ABOVE BY THIS

% bibliography with mutiple entries
%\usepackage{multibib}
%\newcites{book,misc}{{Books},{Others}}
%----------------------------------------------------------------------------------
%            content
%----------------------------------------------------------------------------------
\begin{document}
%\begin{CJK*}{UTF8}{gbsn}                          % to typeset your resume in Chinese using CJK
%-----       resume       ---------------------------------------------------------
\makecvtitle

\small{Computer Science Engineer pursued from Guru Nanak Dev Engineering College, Ludhiana. An innovative thinker, initiative taker and multi dimensional professional with exceptional logical and analytical skills in computer-based software. Looking for an opportunity in the field of information technology as computer programmer in a renowned organization.}

\section{Project Experience}


\vspace{6pt}

\begin{itemize}

\item{\cventry{2017}{https://github.com/GreatDevelopers/pbOSM}{Beautifully Designed Map of India}{GreatDevelopers}{}{\vspace{3pt}This project aims at beautiful design map of India with customizing CartoCSS styles, providing various animations in Javascript and 3-D Map. One can view it over https://lab.gdy.club }}

\vspace{6pt}

\item{\cventry{2016}{https://github.com/amisha2016/Certificates}{Certificate Generation System}{6-weeks Project}{}{\vspace{3pt}This application is a portable application used to Generate Certificate for single candidate providing his/her details along with image, As well as for Batch/Number of candidates by simply providing the CSV format file (containing details of every candidate) along with candidate images in a compressed (tar.gz or zip) folder.}}

\vspace{6pt}

\item{\cventry{2016}{https://github.com/amisha2016/osm\_installation\_script}{Installation of OSM with Shell Scripting}{Automation}{}{\vspace{3pt}A long task to make an OpenStreetMap server is reduced to just two lines command. Isn't it interesting? Just run the script and have a cup of coffee:)}}

\end{itemize}

\section{Education}

\vspace{6pt}

\begin{itemize}

\item{\cventry{2014--2018}{Guru Nanak Dev Engineering College}{Bachelor's Degree in Computer Science and Engineering}{83.4\%}{\textit{Punjab, India}}{}}

\item{\cventry{2014}{B. C. M Arya Model Senior Secondary School, Shastri Nagar, Ludhiana}{Higher Secondary Examination}{86.2\%}{\textit{(CBSE Board)}}{}}  % arguments 3 to 6 can be left empty

\item{\cventry{2012}{ST. Thomas Senior Secondary School, CMC Road, Ludhiana}{Matriculation}{9.0 CGPA}{\textit{(CBSE Board)}}{}}

\end{itemize}

\section{Achievements}

\vspace{6pt}

\begin{itemize}

\item Got selected for final round in "TAP B-Plan Competition" for Start-ups held in 2016, Chandigarh.

\vspace{6pt}

\item Won 3rd Prize in "C for Coders" Event held in College(2015).

\vspace{6pt}

\item Developer for GreatDevelopers Group of College.

\vspace{6pt}

\item Participated in Sports, College Technical and Cultural Events.

\vspace{6pt}

\item Participated in "Training in Excellence" Program of Art of Living(2015).


\end{itemize}

\section{Skillset}

\vspace{6pt}
 
\begin{itemize}



\item \textbf{Technologies I worked in:} Python, C++, HTML, CSS, Javascript, CartoCSS, Postgresql, MySQL, CGI, Shell Scripting, WordPress, Reveal-md, LaTeX.

\item \textbf{Tools:} vim editor, git, selenium, VB, make and cmake utility.

\item Linux Adminstration
\end{itemize}

%Testing and Consultancy Cell, Ludhiana
%I worked on the projects like "Automatation in Presentation". I got acquainted with Linux, Github and many framework like Reveal-js, Reveal-md. It was a good experience as a Beta- Tester in Android Application and contribution in FAQ.


% Publications from a BibTeX file without multibib
%  for numerical labels: \renewcommand{\bibliographyitemlabel}{\@biblabel{\arabic{enumiv}}}% CONSIDER MERGING WITH PREAMBLE PART
%  to redefine the heading string ("Publications"): \renewcommand{\refname}{Articles}
\nocite{*}
\bibliographystyle{plain}
\bibliography{publications}                        % 'publications' is the name of a BibTeX file

% Publications from a BibTeX file using the multibib package
%\section{Publications}
%\nocitebook{book1,book2}
%\bibliographystylebook{plain}
%\bibliographybook{publications}                   % 'publications' is the name of a BibTeX file
%\nocitemisc{misc1,misc2,misc3}
%\bibliographystylemisc{plain}
%\bibliographymisc{publications}                   % 'publications' is the name of a BibTeX file

%-----       letter       ---------------------------------------------------------

\end{document}


%% end of file `template.tex'.

